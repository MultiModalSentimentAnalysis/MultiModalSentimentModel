In this project, we examined a multi-modal model for sentiment recognition in a sequence of frames, in order to evaluate some factors' contribution to this task, i.e. text, face, pose, and scene details. Based on the results, we realized that in sentiment recognition a combination of these factors can lead to better outcomes with higher accuracy and recall.
\\
This method's results can help label data in other cases. For instance, sentiment recognition in other languages with a small number of data is challenging, but via this model, many data could be classified and prepared for sentiment recognition training.


%Here you need to summarize what you did and why this is
%important. Do not take the abstract and put it in the past
%tense. Remember, now the reader has (hopefully) read the
%paper, so it is a very different situation from the abstract.
%Try to highlight important results and say the things you really
%want to get across such as high-level statements (e.g.,
%we believe that .... is the right approach to .... Even though
%we only considered LAN, the .... technique should be applicable
%....) You can also formulate next steps if you want.
%Be brief.